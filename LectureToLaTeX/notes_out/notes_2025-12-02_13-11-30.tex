\documentclass[12pt]{article}
\usepackage{amsmath, amssymb}

\usepackage{amsmath}
\usepackage{amssymb}
\begin{document}

\title{Vector Identity Verification}
\author{}
\date{}
\maketitle

\section{Introduction}

The goal is to prove the following vector identity:
\[
\mathcal{S}(\hat{v}) \times \mathcal{S}(\hat{w}) = \hat{v} \times \hat{w}
\]
where \(\mathcal{S}\) represents a particular transformation of the vector under certain assumptions. We begin by defining and expanding the transformation \(\mathcal{S}\).

\section{Definition of the Transformation}

For a vector \(\hat{v}\), the transformation is defined as:
\[
\mathcal{S}(\hat{v}) = \hat{v} + (\hat{v} \cdot \hat{\lambda}) \hat{s}
\]
For another vector \(\hat{w}\), we have:
\[
\mathcal{S}(\hat{w}) = \hat{w} + (\hat{w} \cdot \hat{\lambda}) \hat{s}
\]

\section{Cross Product Expansion}

To prove the identity, we calculate the cross product:
\[
\mathcal{S}(\hat{v}) \times \mathcal{S}(\hat{w}) = \left( \hat{v} + (\hat{v} \cdot \hat{\lambda}) \hat{s} \right) \times \left( \hat{w} + (\hat{w} \cdot \hat{\lambda}) \hat{s} \right)
\]

Expanding the right-hand side using the distributive property of the cross product gives:
\[
\hat{v} \times \hat{w} + \hat{v} \times \left((\hat{w} \cdot \hat{\lambda}) \hat{s}\right) + (\hat{v} \cdot \hat{\lambda}) \hat{s} \times \hat{w} + (\hat{v} \cdot \hat{\lambda}) \hat{s} \times \left((\hat{w} \cdot \hat{\lambda}) \hat{s}\right)
\]

Observing the components:
- \(\hat{v} \times \hat{w}\) is the required term.
- Each of the remaining terms involves \(\hat{s}\) in a way that either cancels out or nullifies due to orthogonality (\(\hat{\lambda} \cdot \hat{s} = 0\)).

Hence, this simplifies back to:
\[
\hat{v} \times \hat{w}
\]

\section{Geometric Interpretation}

Consider the area of the parallelogram \(\Pi\) formed by the vectors:
- \(\text{Area}(\Pi) = h \times b\), where \(h\) is height and \(b\) is base.

Looking at the effective vector pairing:
\[
(\hat{v} \times (\hat{w} \cdot \hat{\lambda}) \hat{s})
\]
The term becomes \((\hat{v} \cdot \hat{s}) (\hat{w} \cdot \hat{\lambda}) \hat{s}\), which effectively reduces to zero since the vectors \(\hat{v}\) and \(\hat{w}\) lie in the plane formed by \(\hat{\lambda}\) and \(\hat{s}\).

\section{Conclusion}

Through these calculations, we have confirmed that:
\[
\mathcal{S}(\hat{v}) \times \mathcal{S}(\hat{w}) = \hat{v} \times \hat{w}
\]
This verifies the original identity as sought.

\end{document}