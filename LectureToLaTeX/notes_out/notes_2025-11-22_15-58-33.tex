\documentclass[12pt]{article}
\usepackage{amsmath, amssymb}

\newtheorem{theorem}{Theorem}
\newtheorem{lemma}{Lemma}
\newtheorem{definition}{Definition}

\usepackage{amsmath}
\usepackage{amssymb}
\begin{document}

\section{Group Actions and Fixed Fields}

Let us consider a Galois extension \( K/F \) with Galois group \( G = \operatorname{Gal}(K/F) \). The fundamental theorem of Galois theory establishes a one-to-one correspondence between intermediate fields \( E \) (with \( F \subseteq E \subseteq K \)) and subgroups \( H \leq G \). This correspondence is given by:
\[
E \mapsto \operatorname{Gal}(K/E) \quad \text{and} \quad H \mapsto K^H,
\]
where \( K^H = \{ x \in K \mid \sigma(x) = x \ \forall \sigma \in H \} \) denotes the fixed field of \( H \).

\subsection{The Galois Correspondence for Normal Subgroups}

Suppose \( H \trianglelefteq G \) is a normal subgroup. Then the fixed field \( E = K^H \) is a Galois extension of \( F \), and we have:
\[
\operatorname{Gal}(E/F) \cong G/H.
\]
This isomorphism is natural and reflects the fact that the quotient group \( G/H \) acts faithfully on \( E \).

\subsection{A Key Lemma on Group Actions}

We now prove an important lemma concerning the action of \( G \) on \( K \).

\begin{lemma}
Let \( G \) act on \( K \), and let \( H \leq G \). For any \( a \in K \), the following are equivalent:
\begin{enumerate}
    \item \( a \in K^H \), i.e., \( h(a) = a \) for all \( h \in H \).
    \item The orbit \( G \cdot a \) is contained in \( K^H \).
\end{enumerate}
\end{lemma}

\begin{proof}
We prove both directions.

\paragraph{\( (1) \Rightarrow (2) \):} Assume \( a \in K^H \). Let \( g \in G \) and \( h \in H \). Since \( H \) is normal in \( G \) (in the context of Galois extensions, though the lemma holds more generally), we have \( h g = g h' \) for some \( h' \in H \). Then:
\[
h(g(a)) = (h g)(a) = (g h')(a) = g(h'(a)) = g(a),
\]
since \( h'(a) = a \). Thus \( g(a) \in K^H \) for all \( g \in G \), so \( G \cdot a \subseteq K^H \).

\paragraph{\( (2) \Rightarrow (1) \):} Assume \( G \cdot a \subseteq K^H \). In particular, taking \( g = e \) (the identity), we have \( a \in K^H \). More formally, for any \( h \in H \), \( h(a) \in G \cdot a \subseteq K^H \), so \( h(a) \) is fixed by all elements of \( H \). But then \( h(a) = h'(h(a)) \) for all \( h' \in H \), which implies \( a = h(a) \) by applying \( h^{-1} \). Hence \( a \in K^H \).
\end{proof}

\subsection{An Application to Field Elements}

Let us apply this lemma in the context of Galois theory. Suppose \( K/F \) is Galois with group \( G \), and \( H \trianglelefteq G \). Let \( E = K^H \). For any \( a \in K \), consider the following:

Since \( H \trianglelefteq G \), for any \( g \in G \) and \( h \in H \), there exists \( h' \in H \) such that \( g h = h' g \). Then:
\[
h(g(a)) = (h g)(a) = (h' g)(a) = h'(g(a)).
\]
If \( g(a) \in E \), then \( h'(g(a)) = g(a) \), so \( h(g(a)) = g(a) \). This shows that \( g(a) \) is fixed by \( H \), hence \( g(a) \in E \). Therefore, \( G \) permutes the elements of \( E \).

Moreover, if \( a \in E \), then for any \( g \in G \), we have \( g(a) \in E \) as above. This implies that the minimal polynomial of \( a \) over \( F \) splits completely in \( E \), which is consistent with \( E/F \) being Galois.

\subsection{Conclusion}

The interplay between group actions and field theory is fundamental in Galois theory. The lemma proved above illustrates how the structure of the Galois group \( G \) and its subgroups controls the behavior of field elements under automorphisms. This leads to the powerful correspondence between intermediate fields and subgroups, which is the cornerstone of Galois theory.

\end{document}