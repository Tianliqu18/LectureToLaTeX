\documentclass[12pt]{article}
\usepackage{amsmath, amssymb}

\newtheorem{theorem}{Theorem}
\newtheorem{definition}{Definition}
\newtheorem{remark}{Remark}

\usepackage{amsmath}
\usepackage{amssymb}
\begin{document}

\title{Notes on Cyclotomic Fields and Galois Theory}
\author{Blackboard Transcription}
\date{}
\maketitle

\section{Introduction to Cyclotomic Fields}

We begin by considering cyclotomic fields, which are extensions of the rational numbers obtained by adjoining roots of unity. These fields play a fundamental role in algebraic number theory and have deep connections to Galois theory.

\section{Galois Group of Cyclotomic Extensions}

\begin{definition}
Let $\zeta_n$ denote a primitive $n$th root of unity. The $n$th cyclotomic field is $\mathbb{Q}(\zeta_n)$.
\end{definition}

\begin{theorem}
The Galois group of the extension $\mathbb{Q}(\zeta_n)/\mathbb{Q}$ is isomorphic to $(\mathbb{Z}/n\mathbb{Z})^\times$, the multiplicative group of units modulo $n$.
\end{theorem}

Let us examine this isomorphism more carefully. For any integer $a$ coprime to $n$, there is an automorphism $\sigma_a$ of $\mathbb{Q}(\zeta_n)$ defined by:
\[
\sigma_a(\zeta_n) = \zeta_n^a.
\]
This assignment gives the isomorphism:
\[
\operatorname{Gal}(\mathbb{Q}(\zeta_n)/\mathbb{Q}) \cong (\mathbb{Z}/n\mathbb{Z})^\times.
\]

\section{Subfields and Fixed Fields}

An important aspect of Galois theory is the correspondence between subgroups of the Galois group and intermediate fields. For cyclotomic fields, this correspondence is particularly explicit.

Consider the subgroup $H \leq (\mathbb{Z}/n\mathbb{Z})^\times$. The fixed field of the corresponding subgroup of $\operatorname{Gal}(\mathbb{Q}(\zeta_n)/\mathbb{Q})$ is:
\[
\mathbb{Q}(\zeta_n)^H = \{ x \in \mathbb{Q}(\zeta_n) : \sigma(x) = x \text{ for all } \sigma \in H \}.
\]

\subsection{Example: Quadratic Subfields}

When $n$ is an odd prime $p$, the group $(\mathbb{Z}/p\mathbb{Z})^\times$ is cyclic of order $p-1$. If $p-1$ is even, then there is a unique subgroup of index 2. The corresponding fixed field is a quadratic extension of $\mathbb{Q}$.

In fact, one can show that:
\[
\mathbb{Q}(\sqrt{p^*}) \subset \mathbb{Q}(\zeta_p),
\]
where $p^* = (-1)^{\frac{p-1}{2}} p$. This quadratic field is fixed by the unique subgroup of index 2 in $(\mathbb{Z}/p\mathbb{Z})^\times$.

\section{Ramification in Cyclotomic Fields}

The study of how primes factor in cyclotomic extensions is central to algebraic number theory. The following result is fundamental:

\begin{theorem}
Let $p$ be a prime number and $n$ a positive integer. The prime $p$ is ramified in $\mathbb{Q}(\zeta_n)$ if and only if $p$ divides $n$.
\end{theorem}

More precisely, if $n = p^e m$ with $p \nmid m$, then the ramification index of $p$ in $\mathbb{Q}(\zeta_n)$ is $\varphi(p^e) = p^{e-1}(p-1)$.

\section{Discriminant Calculations}

The discriminant of a number field provides important information about its arithmetic properties. For cyclotomic fields, we have:

\begin{theorem}
The discriminant of $\mathbb{Q}(\zeta_n)$ is given by:
\[
\Delta_{\mathbb{Q}(\zeta_n)} = (-1)^{\varphi(n)/2} \frac{n^{\varphi(n)}}{\prod_{p\mid n} p^{\varphi(n)/(p-1)}}.
\]
\end{theorem}

This formula reveals that the primes dividing the discriminant are exactly those dividing $n$, consistent with the ramification theorem above.

\section{Units and the Cyclotomic Unit Group}

The unit group of cyclotomic fields has a rich structure. Of particular interest are the \emph{cyclotomic units}, which generate a subgroup of finite index in the full unit group.

For $\mathbb{Q}(\zeta_p)$ with $p$ an odd prime, the cyclotomic units are defined as:
\[
\eta_a = \frac{\zeta_p^a - 1}{\zeta_p - 1}, \quad \text{for } a = 2, 3, \dots, \frac{p-1}{2}.
\]

These units satisfy various norm relations and play a crucial role in the proof of the Kronecker-Weber theorem and in Iwasawa theory.

\section{Class Number Formulas}

The class number of cyclotomic fields, which measures the failure of unique factorization, can be related to special values of $L$-functions. For $\mathbb{Q}(\zeta_p)$, we have:

\[
h_p = \frac{\prod_{\chi \text{ odd}} (- \frac{1}{2} B_{1,\chi})}{2^{\frac{p-3}{2}} R_p},
\]
where $B_{1,\chi}$ are generalized Bernoulli numbers and $R_p$ is the regulator.

This formula connects the arithmetic of cyclotomic fields to analytic objects, illustrating the deep unity between number theory and analysis.

\section{Conclusion}

Cyclotomic fields provide a beautiful testing ground for many concepts in algebraic number theory. Their explicit Galois groups, well-understood ramification behavior, and connections to special values of $L$-functions make them indispensable in modern number theory.

Further study leads to class field theory, Iwasawa theory, and the theory of $p$-adic $L$-functions, where cyclotomic fields continue to play a central role.

\end{document}