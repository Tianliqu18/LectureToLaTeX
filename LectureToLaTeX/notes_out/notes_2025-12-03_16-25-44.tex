\documentclass[12pt]{article}
\usepackage{amsmath, amssymb}

\usepackage{amsmath}
\usepackage{amssymb}
\begin{document}

\section*{Analysis of the Integral}

In this document, we analyze the following improper integral:

\[
\int_{-\infty}^{\infty} e^{x^2} \, dx
\]

\subsection*{Understanding the Integral}

The integral above represents the integration of the function \( e^{x^2} \) with respect to \( x \) over the entire real line. Since the limits of integration are from \(-\infty\) to \( \infty \), this is an improper integral.

\subsection*{Challenges in Evaluation}

One of the main challenges in evaluating this integral directly is the presence of the function \( e^{x^2} \). The exponential of a square function does not have a simple antiderivative that can be expressed in terms of elementary functions. This complicates the integration process.

\subsection*{Possible Approaches}

To handle integrals of this nature, one typically resorts to special techniques or functions. For instance, a similar integral involving the Gaussian function \( e^{-x^2} \) can be evaluated using polar coordinates, leading to the well-known result:

\[
\int_{-\infty}^{\infty} e^{-x^2} \, dx = \sqrt{\pi}
\]

In contrast, the integral of \( e^{x^2} \) diverges as \( x \rightarrow \pm \infty \), making it improper for usual evaluation techniques without additional context or constraints (such as a factor ensuring convergence).

\end{document}