\documentclass[12pt]{article}
\usepackage{amsmath, amssymb}

\usepackage{amsmath}
\usepackage{amssymb}
\begin{document}

\section{Vector Spaces and Quotients}

In linear algebra, a subspace \( U \) of a vector space \( V \) can be used to construct a new vector space called the quotient space, denoted \( V/U \).

\subsection{Construction of the Quotient Space}

\begin{itemize}
    \item \( U \subseteq V \): Here, \( U \) is a subspace of \( V \).
\end{itemize}

The equivalence relation on \( V \) is defined by:
\[
v \sim v' \quad \text{if} \quad v - v' \in U
\]
where \( v, v' \in V \).

The set of equivalence classes under this relation is:
\[
V/U = \{ [v] \mid v \in V \}
\]
This construction forms a new vector space called the \textit{quotient space}.

\subsection{Bases and Dimension}

\begin{itemize}
    \item Let \( \{ u_1, u_2, \ldots, u_k \} \) be a basis of \( U \).
    \item Extend this basis to a basis of \( V \), denoted \( \{ u_1, \ldots, u_k, v_1, \ldots, v_{\ell} \} \).
\end{itemize}

Then,
\[
\{ [v_i] \mid 1 \leq i \leq \ell \}
\]
is a basis for the quotient space \( V/U \).

The dimension of the quotient space is given by the formula:
\[
\dim V/U = \ell = \dim V - \dim U
\]

This shows that the dimension of the quotient space \( V/U \) is the difference between the dimension of \( V \) and the dimension of \( U \).

\end{document}