\documentclass[12pt]{article}
\usepackage{amsmath, amssymb}

\newtheorem{theorem}{Theorem}
\newtheorem{definition}{Definition}
\newtheorem{lemma}{Lemma}

\usepackage{amsmath}
\usepackage{amssymb}
\begin{document}

\section{Group Actions and Fixed Fields}

Let us consider a Galois extension \( K/F \) with Galois group \( G = \operatorname{Gal}(K/F) \). In this section, we explore the correspondence between subgroups of \( G \) and intermediate fields of the extension \( K/F \).

\subsection{The Fixed Field Correspondence}

Given a subgroup \( H \leq G \), we define the fixed field of \( H \) as:
\[
K^H = \{ x \in K \mid \sigma(x) = x \text{ for all } \sigma \in H \}.
\]
This is a subfield of \( K \) containing \( F \). Conversely, for an intermediate field \( F \subseteq L \subseteq K \), we define the Galois group of \( K \) over \( L \) as:
\[
\operatorname{Gal}(K/L) = \{ \sigma \in G \mid \sigma(x) = x \text{ for all } x \in L \}.
\]
This is a subgroup of \( G \). The fundamental theorem of Galois theory establishes a bijection between subgroups of \( G \) and intermediate fields of \( K/F \).

\subsection{An Important Lemma on Subgroups and Fixed Fields}

We now prove a key lemma relating the fixed fields of conjugate subgroups.

\begin{lemma}
Let \( K/F \) be a Galois extension with Galois group \( G \), and let \( H \leq G \). For any \( \sigma \in G \), we have:
\[
K^{\sigma H \sigma^{-1}} = \sigma(K^H).
\]
\end{lemma}

\begin{proof}
We will show both inclusions.

\paragraph{Proof of \( K^{\sigma H \sigma^{-1}} \subseteq \sigma(K^H) \):}
Let \( x \in K^{\sigma H \sigma^{-1}} \). Then for all \( \tau \in H \), we have:
\[
(\sigma \tau \sigma^{-1})(x) = x.
\]
Applying \( \sigma^{-1} \) to both sides yields:
\[
\tau(\sigma^{-1}(x)) = \sigma^{-1}(x).
\]
This shows that \( \sigma^{-1}(x) \in K^H \), since it is fixed by every \( \tau \in H \). Therefore, \( x = \sigma(\sigma^{-1}(x)) \in \sigma(K^H) \).

\paragraph{Proof of \( \sigma(K^H) \subseteq K^{\sigma H \sigma^{-1}} \):}
Let \( y \in K^H \), and consider \( x = \sigma(y) \). For any \( \tau \in H \), we compute:
\[
(\sigma \tau \sigma^{-1})(x) = (\sigma \tau \sigma^{-1})(\sigma(y)) = \sigma(\tau(y)) = \sigma(y) = x,
\]
since \( \tau(y) = y \) as \( y \in K^H \). Hence, \( x \in K^{\sigma H \sigma^{-1}} \), proving the inclusion.

Therefore, \( K^{\sigma H \sigma^{-1}} = \sigma(K^H) \), as required.
\end{proof}

\subsection{Consequences for Normal Subgroups}

An important special case occurs when \( H \) is a normal subgroup of \( G \).

\begin{theorem}
If \( H \trianglelefteq G \), then the fixed field \( K^H \) is a Galois extension of \( F \), and \( \operatorname{Gal}(K^H/F) \cong G/H \).
\end{theorem}

\begin{proof}
Since \( H \trianglelefteq G \), we have \( \sigma H \sigma^{-1} = H \) for all \( \sigma \in G \). By the previous lemma, this implies:
\[
\sigma(K^H) = K^H \quad \text{for all } \sigma \in G.
\]
This means that \( K^H \) is invariant under the action of \( G \), which implies that \( K^H/F \) is a Galois extension. The restriction map:
\[
\sigma \mapsto \sigma|_{K^H}
\]
defines a surjective homomorphism from \( G \) to \( \operatorname{Gal}(K^H/F) \) with kernel \( H \), so by the first isomorphism theorem:
\[
\operatorname{Gal}(K^H/F) \cong G/H.
\]
\end{proof}

This completes our discussion of the fundamental correspondence in Galois theory between subgroups and intermediate fields, with special attention to the behavior under conjugation and normality.

\end{document}