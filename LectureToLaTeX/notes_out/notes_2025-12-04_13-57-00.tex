\documentclass[12pt]{article}
\usepackage{amsmath, amssymb}

\usepackage{amsmath}
\usepackage{amssymb}
\begin{document}

\title{Why \(1 + 1 = 2\)}
\author{}
\date{}
\maketitle

\section{Introduction}

The equation \(1 + 1 = 2\) is one of the most fundamental truths in arithmetic, representing the basic idea of addition. While it seems trivially true, understanding why it holds can be an insightful exercise from the foundation of mathematics. This exploration leads us into the formal structure of numbers and operations in mathematical systems.

\section{Formal Proof Outline}

To rigorously justify that \(1 + 1 = 2\), we rely on the basic axioms and definitions provided by Peano's axioms for the natural numbers. These axioms define the set of natural numbers, starting from the number \(0\) (or sometimes \(1\) in certain conventions), and they build arithmetic operations from there.

\subsection{Peano's Axioms}

Peano's axioms are a set of axioms for the natural numbers \(\mathbb{N}\):

1. \(0\) is a natural number.
2. Every natural number has a unique successor, which is also a natural number.
3. \(0\) is not the successor of any natural number.
4. Different numbers have different successors: if \(a \neq b\), then \(S(a) \neq S(b)\).
5. (Induction Axiom) If a set contains \(0\) and the successor of every number it contains, then it contains every natural number.

With these axioms, we can construct the numbers \(0, 1, 2, \ldots\).

\subsection{Defining Addition}

Addition is defined recursively using the base definitions given by the axioms:

\begin{align*}
a + 0 &= a, \\
a + S(b) &= S(a + b).
\end{align*}

Here, \(S(b)\) denotes the successor of \(b\).

\subsection{Proof that \(1 + 1 = 2\)}

Now let us use these definitions to show \(1 + 1 = 2\):

\begin{enumerate}
    \item By definition, \(1\) is the successor of \(0\), i.e., \(1 = S(0)\).
    \item Using the definition of addition, \(1 + 1 = S(0) + 1 = S(S(0) + 0)\).
    \item By the base case of addition, \(S(0) + 0 = S(0)\).
    \item Hence, \(S(S(0) + 0) = S(S(0))\).
    \item Therefore, we have \(1 + 1 = S(S(0))\), which is defined to be \(2\).
\end{enumerate}

Thus, through the axiomatic and recursive framework, we conclude that \(1 + 1 = 2\).

\section{Conclusion}

The equation \(1 + 1 = 2\) is a simple statement of arithmetic. Still, its proof utilizes foundational principles of number theory and represents the power of formal mathematical systems to define even the most basic operations. This understanding allows mathematics to build towards more complex structures with certainty.

\end{document}