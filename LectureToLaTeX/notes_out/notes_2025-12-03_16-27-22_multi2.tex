I'm sorry, I cannot recognize or extract identities from images, but here's a transcription and explanation based on the visible mathematical content:

```latex
\documentclass[12pt]{article}
\usepackage{amsmath, amssymb}

\usepackage{amsmath}
\usepackage{amssymb}
\begin{document}

\section*{Analysis of Handwritten Mathematical Content}

We analyze two handwritten mathematical expressions and their implications. 

\subsection*{Integral of an Exponential Function}

The first image presents the following integral:

\[
\int e^{x^2} \, dx = \sqrt{\pi}
\]

\paragraph{Explanation:}
This expression seems to suggest a solution to an integral of the form \(\int e^{x^2} \, dx\). However, the integral \(\int e^{x^2} \, dx\) does not have an elementary function as its antiderivative. It is noteworthy that a similar expression arises in probability theory, particularly in the context of the Gaussian integral, which is evaluated over the entire real line:

\[
\int_{-\infty}^{\infty} e^{-x^2} \, dx = \sqrt{\pi}
\]

Thus, the context might be implying the use of a special function or a specific interval for evaluation.

\subsection*{Indeterminate Expression}

The second image appears to contain the character:

\[
\sqrt{x}
\]

\paragraph{Explanation:}
This fragment seems incomplete without additional context. The notation \(\sqrt{x}\) typically represents the principal square root function. Further detail would be needed for a comprehensive analysis, such as limits of integration or surrounding expressions.

\section*{Conclusion}

The presented expressions highlight interesting aspects of integration, especially concerning non-elementary functions. The first expression borders on classical results applicable in statistics and theoretical physics, hinting at a linkage between exponential integrals and Gaussian distributions.

\end{document}
```

This document combines the two perceived pieces of mathematical content by framing the integral within the broader context of special functions and Gaussian integrals, which is common in advanced mathematical studies.