\documentclass[12pt]{article}
\usepackage{amsmath, amssymb}

\newtheorem{theorem}{Theorem}
\newtheorem{definition}{Definition}
\newtheorem{lemma}{Lemma}

\usepackage{amsmath}
\usepackage{amssymb}
\begin{document}

\title{Group Theory: Normal Subgroups and Quotient Groups}
\author{}
\date{}
\maketitle

\section{Introduction}

In this note, we explore fundamental concepts in group theory, particularly focusing on normal subgroups and quotient groups. The key idea is to understand when a subgroup \( K \) of a group \( G \) is normal, denoted \( K \triangleleft G \), and how this relates to the structure of the quotient group \( G/K \).

\section{Preliminaries}

Let \( G \) be a group and \( K \) a subgroup of \( G \). Recall that \( K \) is normal in \( G \) if and only if for every \( g \in G \) and \( k \in K \), we have \( gkg^{-1} \in K \). Equivalently, \( K \) is normal if \( gK = Kg \) for all \( g \in G \), meaning every left coset is also a right coset.

\section{Characterization of Normal Subgroups}

\begin{theorem}
Let \( G \) be a group and \( K \leq G \). Then the following are equivalent:
\begin{enumerate}
\item \( K \triangleleft G \)
\item For all \( a \in G \), \( aKa^{-1} \subseteq K \)
\item For all \( a \in G \), \( aKa^{-1} = K \)
\end{enumerate}
\end{theorem}

\begin{proof}
We will prove the equivalence of these statements.

First, assume \( K \triangleleft G \). Then for any \( a \in G \) and \( k \in K \), we have \( aka^{-1} \in K \) by definition of normality. This shows \( aKa^{-1} \subseteq K \).

Now assume \( aKa^{-1} \subseteq K \) for all \( a \in G \). To show equality, we also need \( K \subseteq aKa^{-1} \). But this follows by applying the assumption to \( a^{-1} \): \( a^{-1}K(a^{-1})^{-1} = a^{-1}Ka \subseteq K \), which implies \( K \subseteq aKa^{-1} \). Therefore \( aKa^{-1} = K \).

Finally, if \( aKa^{-1} = K \) for all \( a \in G \), then clearly \( K \triangleleft G \) by definition.
\end{proof}

\section{Quotient Groups and the Natural Projection}

Let \( G \) be a group and \( K \triangleleft G \). The quotient group \( G/K \) consists of the cosets of \( K \) in \( G \) with the group operation defined by \( (aK)(bK) = (ab)K \).

There is a natural projection homomorphism \( \pi: G \to G/K \) defined by \( \pi(g) = gK \). This homomorphism is surjective and has kernel \( \ker(\pi) = K \).

\section{Key Observations}

Let's examine some important properties:

If \( K \triangleleft G \), then for any \( a \in G \) and \( k \in K \), we have \( ak = ka' \) for some \( a' \in K \). This follows from the fact that \( aK = Ka \).

Also, if \( k \in K \) and \( a \in G \), then \( aka^{-1} \in K \) by normality. This implies that \( \pi(aka^{-1}) = \pi(k) = K \), the identity element in \( G/K \).

\section{Conclusion}

The condition \( K \triangleleft G \) is fundamental in group theory as it ensures that the quotient \( G/K \) has a well-defined group structure. The equivalence between various characterizations of normality provides multiple perspectives for understanding and verifying this important property.

\end{document}