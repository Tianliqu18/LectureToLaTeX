\documentclass[12pt]{article}
\usepackage{amsmath, amssymb}

\usepackage{amsmath}
\usepackage{amssymb}
\begin{document}

\section{Quotients in Vector Spaces}

Let \( U \subseteq V \) be a subspace.

\subsection{Equivalence Relation}

We define an equivalence relation \(\sim\) on \( V \) by declaring \( v \sim v' \) if \( v - v' \in U \).

\subsection{The Quotient Space \( V/U \)}

The set of equivalence classes is denoted by

\[
V/U = \{ [v] \mid v \in V \}
\]

This creates a new vector space, the quotient space, where each element is a coset \( [v] \).

\subsection{Basis and Dimension}

Suppose \(\{ u_i \}\) is a basis of \( U \), and \(\{ u_1, \ldots, u_k, v_1, \ldots, v_\ell \}\) is a basis of \( V \).

Then, the set

\[
\{ [v_i] \mid 1 \leq i \leq \ell \}
\]

forms a basis of the quotient space \( V/U \).

Therefore, the dimension of the quotient space is given by

\[
\dim V/U = \ell = \dim V - \dim U
\]

This result shows that the dimension of the quotient space is equal to the dimension of the original space minus the dimension of the subspace.

\end{document}