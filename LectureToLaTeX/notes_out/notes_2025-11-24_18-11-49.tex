\documentclass[12pt]{article}
\usepackage{amsmath, amssymb}

\usepackage{amsmath}
\usepackage{amssymb}
\begin{document}

\section*{Group Theory and Normal Subgroups}

In this document, we explore the relationship between groups and their quotient groups with a focus on normal subgroups. Let \( G \) be a group and \( H \) a normal subgroup of \( G \). The quotient group \( G/H \) is defined, and we delve into certain properties and equivalences involving subgroup \( K \).

\section*{Equivalence of Subgroup Conditions}

We wish to establish the equivalence of the two conditions 
\[
\pi(K) = K / (K \cap H)
\]
and 
\[
K / H \trianglelefteq G / H.
\]

\subsection*{Forward Implication (\(\Rightarrow\))}

Assume \(\pi(K) = K / (K \cap H)\).

\begin{itemize}
    \item Consider the projection \(\pi: G \to G/H\).
    \item Let us explore the properties of this map \(\pi\).
    \item Here, \(\pi(g)\pi(k)\pi(g^{-1}) = \pi(gkg^{-1}) = \pi(k')\).
\end{itemize}

This implies:
\[
gkg^{-1} = k'h, \quad h \in H
\]
hence
\[
K \trianglelefteq G.
\]

This follows because \( gKg^{-1} \subseteq K \Rightarrow gKg^{-1} \in K \). Thus \( K \trianglelefteq G \).

\subsection*{Reverse Implication (\(\Leftarrow\))}

Conversely, assume \( K \trianglelefteq G \).

\begin{itemize}
    \item This means \( gKg^{-1} = gKg^{-1}h \) for \( h \in H \).
    \item Thus, \(\pi(g)\pi(k)\pi(g^{-1}) = \pi(gkg^{-1}) = \pi(k')\).
\end{itemize}

We conclude:
\[
K \trianglelefteq K \cap \ker \pi
\]
and therefore
\[
K \subseteq K \cap H \quad \text{so then} \quad \pi(K) = K/H \quad \text{which implies} \quad K/H \trianglelefteq G/H.
\]

\end{document}