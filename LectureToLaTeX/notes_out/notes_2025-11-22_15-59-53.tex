\documentclass[12pt]{article}
\usepackage{amsmath, amssymb}

\newtheorem{theorem}{Theorem}
\newtheorem{definition}{Definition}
\newtheorem{lemma}{Lemma}

\usepackage{amsmath}
\usepackage{amssymb}
\begin{document}

\section{Group Actions and Fixed Fields}

Let us consider a Galois extension \( K/F \) with Galois group \( G = \operatorname{Gal}(K/F) \). In this section, we explore the correspondence between subgroups of \( G \) and intermediate fields of the extension \( K/F \).

\subsection{The Fixed Field Correspondence}

Given a subgroup \( H \leq G \), we define the fixed field of \( H \) as:
\[
K^H = \{ x \in K \mid \sigma(x) = x \text{ for all } \sigma \in H \}.
\]
This is indeed a subfield of \( K \) containing \( F \). Conversely, for an intermediate field \( F \subseteq L \subseteq K \), we define the Galois group of \( K \) over \( L \) as:
\[
\operatorname{Gal}(K/L) = \{ \sigma \in G \mid \sigma(x) = x \text{ for all } x \in L \}.
\]
This is a subgroup of \( G \).

\subsection{Key Properties of the Correspondence}

We now prove an important property relating these constructions.

\begin{lemma}
Let \( H \leq G \) and \( L = K^H \). Then \( \operatorname{Gal}(K/L) = H \).
\end{lemma}

\begin{proof}
By definition, \( H \leq \operatorname{Gal}(K/L) \) since every element of \( H \) fixes \( L \). To show equality, suppose \( \sigma \in \operatorname{Gal}(K/L) \). Then \( \sigma \) fixes \( L \), but since \( L = K^H \), this means \( \sigma \in H \). Thus \( \operatorname{Gal}(K/L) \subseteq H \), and we conclude \( \operatorname{Gal}(K/L) = H \).
\end{proof}

\subsection{Action on Cosets and Intermediate Fields}

Now consider an element \( a \in K \) and its orbit under the action of \( G \). Let \( H = \operatorname{Gal}(K/F(a)) \), the subgroup fixing \( F(a) \). Then the orbit of \( a \) under \( G \) is in bijection with the set of left cosets \( G/H \).

For any \( \sigma \in G \), the element \( \sigma(a) \) is a conjugate of \( a \) over \( F \). The stabilizer of \( a \) in \( G \) is exactly \( H \), so the orbit-stabilizer theorem gives:
\[
|G \cdot a| = [G : H] = [F(a) : F],
\]
where the last equality follows from the Galois correspondence.

\subsection{Normal Subgroups and Galois Subextensions}

An important special case occurs when \( H \) is a normal subgroup of \( G \). In this case, the fixed field \( L = K^H \) is a Galois extension of \( F \), and we have:
\[
\operatorname{Gal}(L/F) \cong G/H.
\]

To see this, consider the restriction homomorphism:
\[
\varphi: G \to \operatorname{Gal}(L/F), \quad \sigma \mapsto \sigma|_L.
\]
The kernel of \( \varphi \) is exactly \( H = \operatorname{Gal}(K/L) \), since these are the automorphisms that act trivially on \( L \). By the first isomorphism theorem, we obtain:
\[
\operatorname{Gal}(L/F) \cong G/H.
\]

\subsection{Application to the Current Context}

In our specific situation, we have \( K/F \) Galois with group \( G \), and we consider an intermediate field \( L \) with \( H = \operatorname{Gal}(K/L) \). The key observation is that for any \( \sigma \in G \), the conjugate field \( \sigma(L) \) corresponds to the conjugate subgroup \( \sigma H \sigma^{-1} \).

In particular, \( L/F \) is Galois if and only if \( H \) is normal in \( G \), in which case:
\[
\operatorname{Gal}(L/F) \cong G/H.
\]

This completes our discussion of the fundamental Galois correspondence between subgroups and intermediate fields.

\end{document}