\documentclass[12pt]{article}
\usepackage{amsmath, amssymb}

\newtheorem{theorem}{Theorem}
\newtheorem{definition}{Definition}
\newtheorem{remark}{Remark}

\usepackage{amsmath}
\usepackage{amssymb}
\begin{document}

\title{Notes on Cyclotomic Extensions and Galois Theory}
\author{Mathematical Blackboard Transcription}
\date{}
\maketitle

\section{Introduction to Cyclotomic Extensions}

These notes explore the structure of cyclotomic extensions and their Galois groups. We begin by recalling the definition of a cyclotomic extension and then proceed to analyze its Galois group structure in detail.

\section{The $n$-th Cyclotomic Extension}

\begin{definition}
Let $n$ be a positive integer. The $n$-th cyclotomic extension of $\mathbb{Q}$ is the field $\mathbb{Q}(\zeta_n)$, where $\zeta_n$ is a primitive $n$-th root of unity.
\end{definition}

The minimal polynomial of $\zeta_n$ over $\mathbb{Q}$ is the $n$-th cyclotomic polynomial $\Phi_n(x)$, which has degree $\varphi(n)$, where $\varphi$ is Euler's totient function.

\section{Galois Group of Cyclotomic Extensions}

\begin{theorem}
The extension $\mathbb{Q}(\zeta_n)/\mathbb{Q}$ is Galois, and its Galois group is isomorphic to $(\mathbb{Z}/n\mathbb{Z})^\times$, the multiplicative group of units modulo $n$.
\end{theorem}

\begin{proof}
Every automorphism $\sigma \in \operatorname{Gal}(\mathbb{Q}(\zeta_n)/\mathbb{Q})$ is determined by its action on $\zeta_n$, and we must have $\sigma(\zeta_n) = \zeta_n^a$ for some integer $a$ coprime to $n$. This gives the isomorphism:
\[
\operatorname{Gal}(\mathbb{Q}(\zeta_n)/\mathbb{Q}) \cong (\mathbb{Z}/n\mathbb{Z})^\times
\]
The isomorphism sends $\sigma$ to the residue class of $a$ modulo $n$.
\end{proof}

\section{Subfields and Fixed Fields}

Let $H$ be a subgroup of $(\mathbb{Z}/n\mathbb{Z})^\times$. Then the fixed field $\mathbb{Q}(\zeta_n)^H$ is a subfield of $\mathbb{Q}(\zeta_n)$.

\begin{remark}
By the Fundamental Theorem of Galois Theory, there is a one-to-one correspondence between subgroups of $(\mathbb{Z}/n\mathbb{Z})^\times$ and subfields of $\mathbb{Q}(\zeta_n)$ containing $\mathbb{Q}$.
\end{remark}

\section{Example: The Case $n = 8$}

Let us consider the specific case where $n = 8$. The primitive 8th roots of unity are $\zeta_8, \zeta_8^3, \zeta_8^5, \zeta_8^7$, where $\zeta_8 = e^{2\pi i/8}$.

The Galois group is:
\[
\operatorname{Gal}(\mathbb{Q}(\zeta_8)/\mathbb{Q}) \cong (\mathbb{Z}/8\mathbb{Z})^\times = \{1, 3, 5, 7\}
\]
This group is isomorphic to $\mathbb{Z}/2\mathbb{Z} \times \mathbb{Z}/2\mathbb{Z}$.

Let us compute the fixed field corresponding to the subgroup $H = \{1, 5\}$. The automorphisms are:
\[
\sigma_1: \zeta_8 \mapsto \zeta_8, \quad \sigma_5: \zeta_8 \mapsto \zeta_8^5
\]

We look for elements fixed by both $\sigma_1$ and $\sigma_5$. Consider:
\[
\alpha = \zeta_8 + \zeta_8^5
\]
Since $\zeta_8^5 = -\zeta_8$, we have:
\[
\alpha = \zeta_8 - \zeta_8 = 0
\]
This is not useful. Instead, consider:
\[
\beta = \zeta_8 + \zeta_8^{-1} = \zeta_8 + \zeta_8^7
\]
Note that $\zeta_8^7 = \zeta_8^{-1}$. Under $\sigma_5$:
\[
\sigma_5(\beta) = \zeta_8^5 + \zeta_8^{35} = \zeta_8^5 + \zeta_8^3
\]
This is not equal to $\beta$.

Let's try a different approach. Consider:
\[
\gamma = \zeta_8 + \zeta_8^3
\]
Under $\sigma_5$:
\[
\sigma_5(\gamma) = \zeta_8^5 + \zeta_8^{15} = \zeta_8^5 + \zeta_8^7
\]
This is different from $\gamma$.

Actually, let's compute the fixed field systematically. The subgroup $H = \{1, 5\}$ has order 2, so the fixed field should have degree $[\mathbb{Q}(\zeta_8):\mathbb{Q}]/|H| = 4/2 = 2$ over $\mathbb{Q}$.

Consider the element:
\[
\theta = \zeta_8 + \zeta_8^5
\]
We already computed that $\zeta_8^5 = -\zeta_8$, so:
\[
\theta = \zeta_8 - \zeta_8 = 0
\]
This is trivial.

Let's try:
\[
\eta = \zeta_8^2
\]
Then $\eta^2 = \zeta_8^4 = -1$, so $\eta = i$ (a primitive 4th root of unity). Under $\sigma_5$:
\[
\sigma_5(\eta) = \sigma_5(\zeta_8^2) = (\zeta_8^5)^2 = \zeta_8^{10} = \zeta_8^2 = \eta
\]
So $\eta$ is fixed by $H$. Therefore, the fixed field is $\mathbb{Q}(i) = \mathbb{Q}(\zeta_4)$.

\section{General Structure}

In general, for a subgroup $H$ of $(\mathbb{Z}/n\mathbb{Z})^\times$, the fixed field $\mathbb{Q}(\zeta_n)^H$ consists of all rational linear combinations of $\zeta_n^a$ that are invariant under all automorphisms in $H$.

If $H$ is the subgroup corresponding to $a \equiv 1 \pmod{m}$ for some $m \mid n$, then the fixed field is $\mathbb{Q}(\zeta_m)$.

\section{Conclusion}

Cyclotomic extensions provide rich examples of abelian Galois extensions over $\mathbb{Q}$. Their Galois groups are well-understood, and the correspondence between subgroups and subfields can be explicitly computed in many cases. The case $n = 8$ illustrates how even simple examples can reveal interesting structure.

\end{document}