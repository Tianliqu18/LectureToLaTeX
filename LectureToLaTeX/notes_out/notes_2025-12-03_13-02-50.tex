\documentclass[12pt]{article}
\usepackage{amsmath, amssymb}

\usepackage{amsmath}
\usepackage{amssymb}
\begin{document}

\section*{Demonstration of a Vector Identity}

We aim to demonstrate the following vector identity involving the transformation \( \mathcal{S} \):

\[
\mathcal{S}(\hat{v}) \times \mathcal{S}(\hat{w}) = \hat{v} \times \hat{w}
\]

\subsection*{Expressions for Transformed Vectors}

Firstly, we write down the expressions for the transformed vectors:

\[
\mathcal{S}(\hat{v}) = \hat{v} + (\hat{v} \cdot \hat{\lambda}) \hat{s}
\]

\[
\mathcal{S}(\hat{w}) = \hat{w} + (\hat{w} \cdot \hat{\lambda}) \hat{s}
\]

Here, \(\hat{\lambda}\) and \(\hat{s}\) appear to be some fixed vectors.

\subsection*{Cross Product Calculations}

We now evaluate the cross product of these transformed vectors:

\[
\begin{align*}
\mathcal{S}(\hat{v}) \times \mathcal{S}(\hat{w}) &= \left( \hat{v} + (\hat{v} \cdot \hat{\lambda}) \hat{s} \right) \times \left( \hat{w} + (\hat{w} \cdot \hat{\lambda}) \hat{s} \right) \\
&= \hat{v} \times \hat{w} + \hat{v} \times ((\hat{w} \cdot \hat{\lambda}) \hat{s}) + ((\hat{v} \cdot \hat{\lambda}) \hat{s}) \times \hat{w} + ((\hat{v} \cdot \hat{\lambda}) \hat{s}) \times ((\hat{w} \cdot \hat{\lambda}) \hat{s})
\end{align*}
\]

\subsection*{Simplifying the Cross Product}

Given the properties of the cross product, and assuming that \(\hat{s} \times \hat{s} = \mathbf{0}\) (since the cross product of any vector with itself is zero), we simplify further:

\[
((\hat{v} \cdot \hat{\lambda}) \hat{s}) \times ((\hat{w} \cdot \hat{\lambda}) \hat{s}) = \mathbf{0}
\]

Also, note that:

\[
\hat{s} \times \hat{s} = \mathbf{0} \quad \text{and} \quad (\hat{v} \cdot \hat{\lambda})(\hat{w} \cdot \hat{\lambda})(\hat{s} \times \hat{s}) = \mathbf{0}
\]

Thus, the equation simplifies to:

\[
\mathcal{S}(\hat{v}) \times \mathcal{S}(\hat{w}) = \hat{v} \times \hat{w}
\]

\subsection*{Geometrical Interpretation}

In the geometrical context of a parallelogram or a related diagram as suggested:

- Consider the plane spanned by \(\hat{v}\) and \(\hat{w}\).
- Any additional terms involving \(\hat{s}\) vanish due to the properties of the cross product.

Thus, we have shown the required vector identity.

\end{document}