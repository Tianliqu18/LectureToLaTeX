\documentclass[12pt]{article}
\usepackage{amsmath, amssymb}

\usepackage{amsmath}
\usepackage{amssymb}
\begin{document}

\title{Cross Product Identity}
\author{}
\date{}
\maketitle

\section{Introduction}

In this document, we aim to demonstrate a vector identity involving a specific operation \( \mathcal{S} \) applied to vectors. The claim to be shown is:

\[
\mathcal{S}(\mathbf{\hat{v}}) \times \mathcal{S}(\mathbf{\hat{w}}) = \mathbf{\hat{v}} \times \mathbf{\hat{w}}
\]

where \( \mathbf{\hat{v}} \) and \( \mathbf{\hat{w}} \) are unit vectors.

\section{Vector Transformation}

We begin by expressing the operation \( \mathcal{S} \) applied to vectors \( \mathbf{\hat{v}} \) and \( \mathbf{\hat{w}} \):

\[
\mathcal{S}(\mathbf{\hat{v}}) = \mathbf{\hat{v}} + (\mathbf{\hat{v}} \cdot \mathbf{\lambda}) \mathbf{s}
\]

\[
\mathcal{S}(\mathbf{\hat{w}}) = \mathbf{\hat{w}} + (\mathbf{\hat{w}} \cdot \mathbf{\lambda}) \mathbf{s}
\]

Here, \( \mathbf{\lambda} \) and \( \mathbf{s} \) are auxiliary vectors, possibly indicating specific directions or scales relevant to the transformation.

\section{Cross Product Computation}

The cross product of the transformed vectors is computed as follows:

\begin{align*}
\mathcal{S}(\mathbf{\hat{v}}) \times \mathcal{S}(\mathbf{\hat{w}}) &= \left( \mathbf{\hat{v}} + (\mathbf{\hat{v}} \cdot \mathbf{\lambda}) \mathbf{s} \right) \times \left( \mathbf{\hat{w}} + (\mathbf{\hat{w}} \cdot \mathbf{\lambda}) \mathbf{s} \right) \\
&= \mathbf{\hat{v}} \times \mathbf{\hat{w}} + \mathbf{\hat{v}} \times \left((\mathbf{\hat{w}} \cdot \mathbf{\lambda}) \mathbf{s}\right) \\
&\quad + \left((\mathbf{\hat{v}} \cdot \mathbf{\lambda}) \mathbf{s}\right) \times \mathbf{\hat{w}} + \left((\mathbf{\hat{v}} \cdot \mathbf{\lambda}) \mathbf{s}\right) \times \left((\mathbf{\hat{w}} \cdot \mathbf{\lambda}) \mathbf{s}\right)
\end{align*}

\section{Simplification}

Observing that the last term evaluates to zero due to the properties of the cross product (\(\mathbf{s} \times \mathbf{s} = \mathbf{0}\)), we focus on simplifying:

\begin{align*}
\mathbf{\hat{v}} \times \left((\mathbf{\hat{w}} \cdot \mathbf{\lambda}) \mathbf{s}\right) &= (\mathbf{\hat{w}} \cdot \mathbf{\lambda}) (\mathbf{\hat{v}} \times \mathbf{s}) \\
\left((\mathbf{\hat{v}} \cdot \mathbf{\lambda}) \mathbf{s}\right) \times \mathbf{\hat{w}} &= -(\mathbf{\hat{v}} \cdot \mathbf{\lambda}) (\mathbf{\hat{w}} \times \mathbf{s})
\end{align*}

Both terms involving \(\mathbf{s}\) cancel each other out under the sum, reaffirming that:

\[
\mathcal{S}(\mathbf{\hat{v}}) \times \mathcal{S}(\mathbf{\hat{w}}) = \mathbf{\hat{v}} \times \mathbf{\hat{w}}
\]

\section{Conclusion}

The manipulation of the expressions confirms the original identity. This property highlights a unique invariance under the transformation \( \mathcal{S} \).

\end{document}